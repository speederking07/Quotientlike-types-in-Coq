In this chapter, we will explore how to define quotient-like types using functions. However, these representations often lead to more problems than solutions. Nonetheless, they are intriguing examples of innovative approaches to quotient types in Coq. These examples are sourced from Coq's set library \cite{coqDoc}. %Better source
\section{Quotient-like functions}
At the beginning of the thesis, we mentioned that it is impossible to define the type of sets and multisets for any arbitrary underlying type. It is true for inductive types, where the order of constructors matters. However, we can utilize the built-in function type. Of course, to reason about the equalities of functions in a sensible manner, we will need to assume the axiom of function extensionality. Introducing this axiom does not introduce any contradictions to the Coq system, so we can safely add it.
\begin{defi}{Function extensionality}{def:FunExt}
\begin{minted}{coq}
Definition FunExt := forall (A B : Type) (f g : A -> B),
    (forall x : A, f x = g x) -> f = g.
\end{minted}
\end{defi}
Without this axiom, two functions are equal if and only if they are intentionally equal, which means they perform computations in the same way. With this axiom, two functions are equal if and only if they give the same results. We can use this property to "forget" the order of elements since the order of computations does not matter as long as it does not change the result.

\section{Sets and multisets}
Knowing this concept, we can define the set as the set characteristic function that decides whether the element is in the set or not.

\begin{defi}{Set as characteristic function}{def:fSet}
\begin{minted}{coq}
Definition set (A : Type) : Type := A -> bool.
\end{minted}
\end{defi}
In contrast to previous definitions of sets, this one lets us define sets with potentially infinitely many elements. We can use this representation to reason about any set with a computable characteristic function. The cost of such a general definition is that some basic functions on set are uncomputable for this representation.
\begin{theo}{Checking emptiness}{th:emptyNotComp}
The function $E : \textrm{set}(X) \rightarrow \{0, 1\}$, which checks whether the set is empty, is uncomputable.
\end{theo}
\begin{proof}{}{proof:emptyNotComp}
Let $E$ be a computable function. For any Turing machine $M$ there exists set $T_M = \{t \in \mathbb{N} : \textrm{after} \; t \; \textrm{steps machine} \; M \; \textrm{is in HALT state}\}$. Every set $T_M$ has a computable characteristic function since checking if the Turing machine halts after a specific number of steps is a computable problem. Checking if $T_M$ is empty is equivalent to checking if the Turning machine ever halts. However, this is an undecidable problem. Therefore, function $E$ is not computable. \phantom{dasd} \contradiction
\end{proof}
Multisets can be defined similarly by replacing codomain type from booleans to natural numbers.
\begin{defi}{Multisets as characteristic function}{def:fMSet}
\begin{minted}{coq}
Definition mset (A : Type) : Type := A -> nat.
\end{minted}
\end{defi}
We can express in this representation only multisets that have finitely many copies of each element of the underlying type. In order to generalize this representation, we need to replace natural numbers with co-natural numbers.
\begin{coq}{Co-natural numbers}{coq:conat}
In Coq, \emph{co-natural numbers} can be defined in the following way:
\begin{minted}{coq}
CoInductive conat : Type := 
| O' : conat
| S' : conat -> conat.
\end{minted}
In contrast to an inductive definition of natural numbers, co-natural numbers are coinductive. That means we define the destructor instead of constructors. Destructor produces from a co-natural number either \mcoq{O'} -- zero or \mcoq{S' n} -- the successor of co-natural number n. Since we use a destructor, we can define a number that is a successor of itself.
\begin{minted}{coq}
CoFixpoint inf : conat := S' inf.
\end{minted}
This number can be called infinity. We can also define any finite number.
\begin{minted}{coq}
Fixpoint toConat (n : nat) : conat :=
  match n with
  | O => O'
  | S n' => S' (toConat n') 
  end.
\end{minted}
\end{coq}
\section{Basic operations}
As we showed in Theorem \ref{th:emptyNotComp}, checking if the set is empty is an undecidable problem. Comparing two sets is an uncomputable operation since we can compare a set to an empty set to check if it is empty. Another missing operation is a mapping operation. We can check emptiness by mapping every element to the element of the unit type if a mapping is computable. Despite those problems, we can still define helpful functions for this set representation. Checking if some element is in a set is trivial since we represent a set as a characteristic function. We can easily define the filtering function.
\begin{func}{Filtering}{fn:set_filter}
\begin{minted}{coq}
Definition set_filter {A : Type} (p : A -> bool) 
  (s : set A) : set A :=
  fun x : A => if p x then s x else false.
\end{minted}
\end{func}
This definition utilizes lazy evaluation. Therefore, potential infinity is not a problem. Similarly, we can define union, intersection, and complement of a set.
\pagebreak
\begin{func}{Union}{fn:set_union}
\begin{minted}{coq}
Definition set_union {A : Type} (s s' : set A) : set A :=
  fun x : A => (s x) || (s' x).
\end{minted}
\end{func}
\begin{func}{Intersection}{fn:set_intersection}
\begin{minted}{coq}
Definition set_intersection {A : Type} (s s' : set A) : set A :=
  fun x : A => (s x) && (s' x).
\end{minted}
\end{func}
\begin{func}{Complement}{fn:set_complement}
\begin{minted}{coq}
Definition set_complement {A : Type} (s : set A) : set A :=
  fun x : A => negb (s x).
\end{minted}
\end{func}
Those operations can also be implemented similarly for multisets using sum and minimum operations. The complement operation is not defined for multisets. The drawback of these lazy computations is that with each subsequent intersection and union, the complexity of the set characteristic function increases. The operation of adding an element to the set can also be defined. However, it requires the underlying type with decidable equality.
\begin{func}{Insertion}{fn:set_add}
\begin{minted}{coq}
Definition set_add {A : Type} `{EqDec A} (a : A) (s : set A)  
  : set A :=
  fun x : A => if eqf x a then true else s x.
\end{minted}
\end{func}
\begin{func}{Conversion to a set}{fn:list_to_set}
\begin{minted}{coq}
Fixpoint list_to_set {A : Type} `{EqDec A} (l : list A) 
  : set A :=
  match l with
  | []        => fun _ => false
  | (h :: l') => set_add h (list_to_set l')
  end.
\end{minted}
\end{func}
Proofs of fundamental properties of those operations can be found in \coqsource{FunctionalQuotient.v}.