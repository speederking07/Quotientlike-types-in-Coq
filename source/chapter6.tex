We wcześniejszych rozdziałach poznaliśmy w jaki sposób możemy zdefiniować typy ilorazowe w Coqu. A mówiąc dokładniej jak możemy w nim obejść problem braku typów ilorazowych. Część rozwiązać nie wymaga użycia typów zależnych i może zostać zaimplementowana w prawie każdym języku programowania , a inne jak te wykorzystujące pod typowanie wymaga aby język wspierał typu zależne od termów. W tym rozdziale skupimy się jednak na językach, które mają mechanizmy pozwalające na bezpośrednią implementację typów ilorazowych, a co za tym idzie można je stosować z dużo większą łatwością.
\section{Lean}
Lean, a bardziej Lean 4, gdyż skupimy się tutaj głównie na tej wersji, jest to asystent dowodzenia rozwijany od 2013 roku. Jest to narzędzie open source pozwalające podobnie jak Coq na dowodzenie poprawności programów, jak i twierdzeń matematycznych. Jest ono częścią programu Microsoft Research.
\subsection{Różnice w stosunku do Coqa}
Lean w przeciwieństwie do Coqa nie wymaga konstruktywności dla twierdzeń\cite{lean4}. Oznacza to, że mamy do czynienia z systemem bardziej zbliżonym do matematycznych dowodów twierdzeń, niż do programowania. Oznacza to niestety, że nie każdy dowód jest w istocie programem, w myśl izomorfizmu Currego-Howarda. W Leanie ponadto mamy definicyjną irrelewację zdań. System formalny opiera się na trzech aksjomatach:
\begin{description}
    \item[Aksjomat ekstensjonalności zdań] - mówi on, że jeśli dwa zdania są sobie równoważne, to są sobie równe.
    \begin{minted}{coq}
Axiom prop_ext: forall P Q: Prop, (P <-> Q) <-> (P = Q)
    \end{minted}
    \item[Aksjomat wyboru] - mówi on, że z każdego niepustego typu możemy wyprodukować jego element.
    \begin{minted}{coq}
Inductive NonEmpty (A: Type) : Prop := intro : A -> NonEmpty A.
Axiom choise: forall A: Type, NonEmpty A -> A.
    \end{minted}
    \item[Aksjomat istnienia ilorazów] - mówi on, że dla każdego typu oraz relacji możemy wyprodukować typ ilorazowy, w którym wszystkie elementy które są ze sobą w tej relacji są sobie równe.
\end{description}
Część z was zapewne wie, że jedną z głównych różnić w stosunku do Coqa jest zastosowanie w Leanie logiki klasycznej. Wydawałoby się zatem, że prawo wyłączonego środka powinno być jednym z aksjomatów w tym systemie. Okazuje się jednak, że wymienione powyżej aksjomaty wystarczą do wyprowadzenia prawa wyłączonego środka jak i egzystencjaliści funkcji. Egzystencjalność można wyprowadzić z istnienia ilorazów, natomiast dowód wyłączonego środka korzysta z konstrukcji zaproponowanej przez Diaconescu w 1975 roku \cite{choise}. Dowody te można również znaleźć w bibliotece standardowej leana pod nazwą \mintinline{lean}{em} oraz \mintinline{lean}{funext}.
\subsection{Typy ilorazowe}
Jak widzieliśmy powyżej, typy ilorazowe są w sercu języka Lean. Znając już główne różnice z Coqiem powinniśmy omówić dokładniej jakie aksjomaty towarzyszą typom ilorazowym w Leanie. 
\begin{code}
\begin{minted}{coq}
Axiom Quot : forall {A: Type}, (A -> A -> Prop) -> Type.
\end{minted}
\caption{Odpowiednik aksjomatu \mintinline{lean}{Quot} w Coqu.}
\label{Quot}
\end{code}

Pierwszy aksjomat \mintinline{lean}{Quot} \ref{Quot} postuluje istnienie typów ilorazowych, tworzonych z dowolnego typu bazowego, oraz dowolnej relacji na tym typie.
\begin{code}
\begin{minted}{coq}
Axiom Quot_mk : forall {A:  Type} (r: A -> A -> Prop),
  A -> Quot r.
\end{minted}
\caption{Odpowiednik aksjomatu \mintinline{lean}{Quot.mk} w Coqu.}
\label{Quot.mk}
\end{code}

Drugi aksjomat \mintinline{lean}{Quot.mk} \ref{Quot.mk} postuluje istnienie funkcji, która tworzy elementy typu ilorazowego.
\begin{code}
\begin{minted}{coq}
Axiom Quot_ind : 
  forall (A:  Type) (r: A -> A -> Prop) (P: Quot r -> Prop),
    (forall a: A, P (Quot_mk r a)) -> forall q: Quot r, P q.
\end{minted}
\caption{Odpowiednik aksjomatu \mintinline{lean}{Quot.ind} w Coqu.}
\label{Quot.ind}
\end{code}

Trzeci aksjomat \mintinline{lean}{Quot.ind} \ref{Quot.ind} postuluje prawa indukcji na typach ilorazowych. Mówią one, że jeśli dla każdego elementu powstałego z elementu typu bazowego zachodzi predykat, to zachodzi on dla całego typu ilorazowego.
\begin{code}
\begin{minted}{coq}
Axiom Quot_lift :
  forall (A:  Type) (r: A -> A -> Prop) (B: Type) (f: A -> B),
    (forall a b: A, r a b -> f a = f b) -> Quot r -> B.

Axiom Quot_lift' : forall {A:  Type} {r: A -> A -> Prop} {B: Type} 
  (f: A -> B) (P: forall a b: A, r a b -> f a = f b) (x: A),
    f x = Quot_lift f P (Quot_mk r x).
\end{minted}
\caption{Odpowiednik aksjomatu \mintinline{lean}{Quot.lift} w Coqu.}
\label{Quot.lift}
\end{code}

Czwarty aksjomat \mintinline{lean}{Quot.lift} \ref{Quot.lift} mówi o tym jak możemy aplikować funkcje, z typu bazowego na elementach typu ilorazowego. Aby taka aplikacja była możliwa wymaga on, aby funkcja szanowała relacje, a więc jeśli dwa elementy są ze sobą w relacji, to wynik funkcji dla nich obu musi być taki sam. W Leanie aksjomat \mintinline{lean}{Quot.lift} przychodzi w parze z regułą przepisywania, która mówi iż rzeczywiście zaaplikowanie funkcji $f$ respektującej relację na elemencie typu ilorazowego daje identyczny efekt, jak zaaplikowanie jej na tym samym elemencie typu pierwotnego. Stad też aby można było skorzystać z typów ilorazowych takich jak w Leanie należy w Coqu zapostulować dodatkowy aksjomat z tą regułą przepisywania. 
\begin{code}
\begin{minted}{coq}
Axiom Quot_sound :
  forall (A: Type) (r: A -> A -> Prop) (a b: A),
    r a b -> Quot_mk r a = Quot_mk r b.
\end{minted}
\caption{Odpowiednik aksjomatu \mintinline{lean}{Quot.sound} w Coqu.}
\label{Quot.sound}
\end{code}

Piąty i ostatni aksjomat \mintinline{lean}{Quot.sound} \ref{Quot.sound} postuluje równość elementów typu ilorazowego, jeśli są one w relacji. Oznacza to, że typy ilorazowe w Leanie rzeczywiście są ilorazowe, czyli sklejone zgodnie z relacją.

Przedstawione powyżej aksjomaty są wystarczające, żeby wprowadzić do języka typy ilorazowe. Możemy je bez problemu również wprowadzić do Coqa i cieszyć się w nim typami ilorazowymi opartych na tych aksjomatach.
\section{Agda}
Agda jest językiem programowania stworzonym z myślą o wsparciu dla typów zależnych \cite{agda}. Postał on jako rozszerzenie teorii typów Martina-Löfa \cite{MARTINLOF197573}. Z uwagi na te cechy może służyć jako asystent dowodzenia, w przeciwieństwie jednak do języków jakich jak Coq czy Lean Agda nie posiada języka taktyk, co znacząco utrudnia wykorzystanie jej w tym celu. Natomiast jej obsługa typów zależnych jest na dużo wyższym poziomie niż to co możemy doświadczyć w Coqu. Potrafi sama w wielu wypadkach wywnioskować, że danych przypadek jest niemożliwy, przez co definiowanie funkcji zależnych jest duża łatwiejsze niż w Coqu.
\begin{code}
\begin{minted}{agda}
lookup : ∀ {A} {n} → Vec A n → Fin n → A
lookup (x ∷ xs) zero    = x
lookup (x ∷ xs) (suc i) = lookup' xs i
\end{minted}
\caption{Definicja funkcji zwracającej $n$-ty element zależnego wektora w Agdzie.}
\label{agda-lookup}
\end{code}
\subsection{Kubiczna Agda}
\emph{Cubical} jest rozszerzeniem do języka Agda, który rozbudowuje możliwości języka o kubiczną teorię typów \cite{cubical}. Pozwala ono na modelowanie konstrukcji z homotopicznej teorii typów natywnie w Agdzie. A więc daje dostęp do takich funkcji jak transport \ref{transport} poznanego w wcześniej. Wprowadza ono koncepcję ścieżek jako dowodów równości między elementami. Wszystkie te własności mogliśmy w większym lub mniejszym stopniu zamodelować w Coqu, tym co wyróżnia kubiczną Agdę na tle innych asystentów dowodzenia, są wyższe typy indykatywne (HIT). Tak jak zwykłe typy indykatywne pozwalają na definiowanie pewnych struktur danych, w przeciwieństwie jednak do znanych nam już typów induktywnych które automatycznie generują równości między elementami, w wyższych typach induktywnych możemy ręcznie dodać kolejne ścieżki. Oznacza to pełną dowolność w definiowaniu równości na definiowanym typie.
\begin{code}
\begin{minted}{agda}
data S : Set where
  base : S
  loop : base = base
\end{minted}
\caption{Definicja okręgu w kubicznej Agdzie.}
\label{agda-bag}
\end{code}
\subsection{Definicja typów ilorazowych za pomocą wyższych typów induktywnych}
Jak możemy zobaczyć definicja multi-zbioru skończonego \ref{agda-bag} jest niezwykle łatwa w kubicznej Agdzie. Wystarczyło dodać dodatkową ścieżkę (dowód równości) na listach, które mają dwa początkowe elementy zamienione kolejnością. Przechodniość równości jest zapewniona z jej definicji, a więc nie musimy jej definiować, natomiast równość dla pustego mulit zbioru oraz następnika dostajemy za darmo, z ich definicji. Ostatecznie więc definicja ta jest krótsza niż klasyczna definicja permutacji w Coqu.
\begin{code}
\begin{minted}{agda}
data Bag (X : Set) : Set where
  nil  : Bag X
  _::_ : X -> Bag X -> Bag X
  swap : (x y : X) (z : Bag X) -> x :: y :: z = y :: x :: z
\end{minted}
\caption{Definicja multi-zbioru skończonego w kubicznej Agdzie.}
\label{agda-bag}
\end{code}

W bardzo podobny sposób możemy zdefiniować liczby całkowite w kubinczej Agdzie, za pomocą zera, następnika oraz poprzednika. Aby zagwarantować jednak izomorfizm tej konstrukcji z znanymi nam z matematyki liczbami całkowitymi musimy dodać dwie dodatkowe równości mówiące w jaki sposób następnik i poprzednik nawzajem się niwelują. Otrzymaliśmy zatem prawie liczby całkowite \ref{agda-int}, ich problemem jest fakt, że zgodnie z naszą definicją istnieje wiele dowodów równości między tymi samymi liczbami całkowitymi, możemy go rozwiązać dodając jeszcze jedną równość tym razem między dowodami równości, ale na potrzeby tej pracy możemy uznać uznać tą definicję za sukces.
\begin{code}
\begin{minted}{agda}
data Int : Set where
  zero  : Int
  succ  : Int -> Int
  pred  : Int -> Int
  ps_eq : (z: Int) -> pred (succ z) = z
  sp_eq : (z: Int) -> succ (pred z) = z
\end{minted}
\caption{Definicja liczb całkowitych w Agdzie.}
\label{agda-int}
\end{code}

Podsumowując wyższe typy indykatywne nadają się doskonale do definiowania typów ilorazowych, niestety nie możemy ich zastosować do definiowania typów ilorazowych w Coqu gdyż jego system typów nie wspiera tak zaawansowanych konstrukcji. Próba odtworzenia dodawania równości między elementami za pomocą aksjomatów bardzo szybko skoczy się wprowadzeniem do systemu sprzeczności, 